\subsection{实现指令流水线的硬布线控制器}
\subsubsection{实现原理}
\par
计算机的流水处理过程非常类似于工厂中的流水装配线,为了实现流水,首先把输入的任务分割为一系列子任务,并使各子任务能在流水线的各阶段并发的执行。当任务连续不断地输入流水线时,在流水线的输入端便连续不断地吐出执行结果,从而实现了子任务的并行性。
\par
用于该次实验的指令集中多为二节拍数指令,分别用于取值和执行,由于\tec 取指用总线和数据交换用总线并不会冲突,因此二节拍指令可将取值和执行并行执行,实现两级流水。三节拍指令仅有LD\comma ST两个指令,在第二三节拍均会使用到数据交换用总线,而\tec 仅具有一条数据交换用总线,为了防止总线占用冲突,可以和这两条指令并行执行的指令只有JC\comma JZ\comma STP三条,若仅对这三条指令采用三级流水效率提升并不高。
\subsubsection{实现方法}
\par 
经过小组组员的共同讨论,我们决定仅将指令周期分为两部分,分别为取指令和执行。
\par
对于一个包含n条指令的程序,若指令全为需要两个节拍数的指令,使用流水线需要n+1个节拍,而使用非流水线需要2n个节拍,当n趋于无穷时,加速比S趋于2;若指令全为需要三个节拍数的指令,使用流水线需要2n+1个节拍数,不使用流水线需要3n个节拍数,当n趋于无穷时,加速比S趋于1.5。
\par 
为了进一步简化实现逻辑,我们在原有的指令集基础上添加了NOP指令,用于第一次进入程序时的取指,后续指令的取值放在前一条指令的最后一个节拍内,实现取指和执行的时间上并行。指令流程图如下:
\begin{figure}[hbt!]
    \centering
    \label{流水线指令流程图}
    \includegraphics[width=\textwidth]{figures/chapter3/流水线指令流程图1.png}
\end{figure}
\newpage
\begin{figure}[hbt!]
    \centering
    \label{流水线指令流程图}
    \includegraphics[width=\textwidth]{figures/chapter3/流水线指令流程图2.png}
    \caption{流水线指令流程图}
\end{figure}
\par
我们依据指令流程图修改了译码表,用于程序组合逻辑的实现,译码表修改部分如下:
\begin{figure}[hbt!]
    \centering
    \label{流水线译码表修改部分}
    \includegraphics[width=\textwidth]{figures/chapter3/流水线译码表修改部分.png}
    \caption{流水线译码表修改部分}
\end{figure}


