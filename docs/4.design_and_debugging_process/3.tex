\subsection{调试过程遇到的问题}
\begin{enumerate}[(1)]
    \item 问题:采用时序逻辑实现信号的输出,并将信号的输出和不需要的信号的清零放在两个不同的时机进行,可能由于试验台的时序信号输出不稳定,导致程序有时能够正常运行,有时无法正常运行。
    \par 解决方案:将信号的输入和清零由时序逻辑修改为组合逻辑,增强了程序的健壮性,成功解决了问题。
    \item 问题:为了实现二级中断,我们将取值和指令执行放在同一节拍电位执行,但这点对于jz和jc指令无法实现,原因时pcinc和pcadd无法无法同时有效。
    \par 解决方案:取消jz和jc指令的流水,将取值和跳转放在两个节拍执行。
    \item 问题:我们发现中断返回时,若仅将寄存器内保存的中断断点写入pc寄存器,程序无法正常进行。原因是指令寄存器内指令依旧为中断返回iret指令,我们并未定义iret指令生效且当前不处于中断时的行为,导致程序无法正常获取下一条指令。
    \par 解决方案:将寄存器内保存的中断断点写入pc寄存器时输出long信号,并在下一个节拍电位w3实现取指,程序能够重新正常运行。
\end{enumerate}