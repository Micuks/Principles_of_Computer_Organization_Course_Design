\subsection{Restatement of Problem}
With basic knowledge of river water modeling forecasting and planning, we analyzed water changes and water reallocation in the Colorado River under the current drought conditions. Our team was assigned to solve the following problems:
\begin{itemize}
  \item[$\circledcirc$] \textbf{Problem 1:} Describe how much water must be exported from Lake Mead and Lake Powell to meet each state's water demand if \textbf{the volume of water} in each lake is M and the volume of water in Lake Powell is P. How long will it take to meet these demands if \textbf{no additional water is considered and water demands are considered to be fixed}? How much \textbf{additional water} must be supplied over time to ensure that \textbf{water demands are met}?
  \item[$\circledcirc$] \textbf{Problem 2:} Build the \textbf{system dynamics model} of Colorado River water resources prediction model over time. Considering the impact of the \textbf{ongoing drought} on water quantity.
  \item[$\circledcirc$] \textbf{Problem 3:} Use the model to solve \textbf{water allocation problem} for agriculture, domestic, industrial and power generation.
  \item[$\circledcirc$] \textbf{Problem 4:} Solving the problem of what to do when there is \textbf{not enough water available for agriculture, domestic, industrial and power generation industries}.
  \item[$\circledcirc$] \textbf{Problem 5:} Answer what happens when \textbf{the demand for population, industry and power generation in drought-affected areas changes}, what happens when \textbf{renewable energy technologies (e.g., hydropower) are more advanced} than the model predicts, and what happens when \textbf{external water sources and power conservation measures} are used.
\end{itemize}