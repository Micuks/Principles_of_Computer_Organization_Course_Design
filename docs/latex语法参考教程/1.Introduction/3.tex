\subsection{Analysis of Problem}
To work out the four problems, our solutions will be proceeded as follows.
\begin{itemize}
  \item[$\circledcirc$] \textbf{Problem 1:} To solve this problem, the first step is to determine the agricultural needs, industrial needs, residential needs and power generation needs of the five states, noting that Lake Mead also requires water from Lake Powell. So Lake Powell \textbf{has one more water demand target} than Lake Mead. After finding the water demands of the five states in recent years, we used \textbf{system dynamics modeling} to develop \textbf{a time-varying target planning model}, and to allocate water from the two lakes to the five states accordingly.
  \item[$\circledcirc$] \textbf{Problem 2,3:} \textbf{Prioritization of water supply} for general (agricultural, industrial, residential) and power generation uses, taking into account the importance of each use in terms of water demand, to address the issue of supply priorities and how to allocate water \textbf{when there is a shortage}.
  \item[$\circledcirc$] \textbf{Problem 4:} The change in water and electricity demand over time is determined by using \textbf{system dynamics methods} based on \textbf{the increase or decrease} in population, agriculture and industry in the area concerned, adjusting the parameters of hydropower generation efficiency according to \textbf{the improvement of renewable energy generation technologies}, and taking into account \textbf{the additional water and electricity savings} in a certain way, and then adjusting the model parameters to obtain the results of the model after changing the parameters.
\end{itemize}
\par
Above is a sketch of the analysis process on the four problems.