\subsection{控制器的设计思路}
硬布线控制器是将控制部件做成产生专门固定时序控制信号的逻辑电路,产生各种控制信号,
因而又称为组合逻辑控制器。这种逻辑电路以使用最少元件和取得最高操作速度为设计目标,
因为该逻辑电路由门电路和触发器构成的复杂树型网络,所以称为硬布线控制器。
\par{硬布线控制器的基本原理}
每个微操作控制信号S是一系列输入量的逻辑函数,用组合逻辑来实现。
\begin{equation}
    S=f(I_m,M_i,T_k,B_j)
\end{equation}
其中,$I_m$是机器指令操作码译码器的输出信号,$M_i$是节拍电位信号,$T_k$是节拍脉冲
信号,$B_j$是状态条件信号。
\par
在\tec 中,节拍脉冲信号$T_k(T1\wave T3)$已经直接送给数据通路。因为机器指令系统
比较简单,省去操作码译码器,4位指令操作码$IR4~IR7$直接成为$I_m$的一部分;由于
\tec 有控制台操作,控制台操作可以看做一些特殊的功能复杂的指令,因此SWC、SWB、SWA
可以看做是$I_m$的另一部分。$M_i$是时序发生器产生的节拍信号$W_1\wave W_3$;
$B_j$包括ALU产生的进位信号C、结果为0的信号Z等等。
\begin{figure}[htbp]
    \label{硬布线控制器逻辑模块图}
    \fixedgraphic{figures/硬布线控制器逻辑模块图.png}
    \caption{硬布线控制器逻辑模块图}
\end{figure}