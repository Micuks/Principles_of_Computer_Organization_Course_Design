\subsection{机器指令的节拍数}
\tec 模型计算机的时序采用不定长机器周期,执行绝大多数指令时,节拍发生器产生节拍电位信号
W1、W2,实现二周期指令的执行。
\par
对于单周期,三周期和四周期指令,我们采用了这样的方式实现:
\begin{itemize}
    \item 引入控制信号SHORT和LONG;
    \item 当SHORT信号发出时,将通知节拍发生器在W1后不再产生W2信号,
    以满足单周期指令的要求;
    \item 而当LONG信号发出时,将通知节拍发生器在W2后再产生一个W3信号,
    以满足三周期指令的需求;
    \item 对于四周期指令,引入标志量$st0$,将两个双节拍电位周期合并为一个四节拍周期。
    当$st0==0$时表示目前处在第一个双节拍电位周期,此周期结束后置$st0=1$,
    表示目前处在第二个双节拍电位周期。\footnote{由于四周期指令使用较少,为节省资源,
    $st0$标志位还承担着标识题目一选做任务2\textbf{修改PC指针功能}的工作,
    将在下面的章节进行介绍。}
\end{itemize}