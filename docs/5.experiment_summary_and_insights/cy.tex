本次课程设计的过程中,我们小组四名同学共同完成了流水线和中断两个版本的硬布线控制器。我们采用verilog语言编写实验程序,从中也遇到了不少问题。
\par
首先是对硬件语言的不熟悉,由于过去软件编写采用的都是顺序执行的语言,在思考程序架构的时候还是采用以前的思维逻辑,将想法付诸实现后多次编译不通过,包括无法在多个always块中对同一个变量赋值,不能同时检测同一个输入的上沿和下沿等。即使编译通过,程序也不一定会像想象中一样执行。
然后就是硬件输出时序逻辑的不稳定,一开始尝试使用时序逻辑实现信号的输出,在仿真软件能够正常执行。但当下载到硬件上后,有时能够正常执行,有时不能正常执行,这一点就让我们很头疼,不确定到底硬件的问题还是代码的问题,即使是代码的问题也不知道究竟是哪里出了问题,在这一个问题上耗费了大量的时间。直到实验已经开始四天后,还是没找到问题关键所在,之能重新修改程序架构,将时序逻辑改成组合逻辑实现才顺利解决问题。
最后就是时间紧张,实验时间满打满算就只有一周的时间,一些想法没有时间付诸实践,包括多级中断的实现等等,这些想法只能放在实验报告中仅供参考。
\par
不过总的来说还是收获很多,了解了硬件语言开发在仿真和实际在硬件上操作的巨大区别,锻炼了团队合作能力,培养了程序的调试能力等等,我相信本次课程设计能为我今后的专业学习以及实际工程开发道路带来启发。同时,老师在实验验收过程中,对于我们实验班同学的批评也改变了我对于以后做实验的一些想法,希望在未来能够有所帮助。
