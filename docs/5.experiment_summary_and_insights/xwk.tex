\subsection*{谢卫凯}

在最开始的理论课程结束后,由于理论课程学习中对硬布线控制器介绍较为抽象,我仍对硬布线控制器的实现感到困惑。后续对
之前实验课程的各种资料进行复习后,重新捡起了TEC-8系统中各个部件如运算器、双端口存储器、各个寄存器的工作方式。熟悉了
TEC-8,再回过头来看硬布线控制器,就已经对大致要实现的功能了然于心了。

经过和队友的商量,我们队伍决定采用更贴近工程实践的Verilog HDL进行设计,我在大致明白了TEC-8的工作方式和硬布线控制器
的功能后,便开始了解学习Verilog HDL。在一天半左右的时间里,我参考了网络上的各种资料,同时将Verilog和以前学习的VHDL
进行对比,很快也差不多掌握了Verilog中的基础语法和各种基本概念。例如wire类型相当于电路中的接线,和VHDL中的signal类似,
reg类型的变量可以看做一个寄存器,always和process一样由一些事件触发,明白了阻塞赋值和非阻塞赋值的区别等等。队伍成员都
大致可以使用Verilog后,我们便开始了设计与实现硬布线控制器的工作。

在编写代码前,我们队伍先设计好了流程图,有了流程图后,我和队友分别尝试采用组合逻辑和行为描述的方式实现基本硬布线控制器,
在第二种实现方式调试时,我们发现试验台总是会产生怪异表现,经过排查,我们认为是由于对硬件理解太浅,编写代码不够规范导致
在电路综合时出现了不稳定的电路。于是全组都转向了采用组合逻辑的实现方式,之后很快就完成了基础版本。

在流水线的实现上,由于TEC-8系统很简单,多数指令再两个节拍就能执行完毕,于是我们采用了在指令执行阶段同时取指的方式完成流水线
设计与实现。这也加深了我对流水线的认识,实际见到了在流水线方式下执行指令和非流水线下执行指令的差别。

在最后的中断设计上,由于硬布线控制器无法控制LIAR等与中断相关的控制信号,我们便使用了R3寄存器一直跟踪PC的值,以此来达到
记录PC并且在中断后能返回到断点位置的功能。在实现了基本的单级中断后,我们的小组成员还提出了利用存储器保存PC值以实现多级中断
的设计思路,但碍于时间有限没能实现。

总的来说,这一周的时间里收获不少。学习了Verilog,对计算机组成中的运算器、存储器、控制器的工作方式有了更深的理解。在这个
过程中我们队伍也遇到了不少困难,在团队的力量下我们也都能克服这些困难,很高兴能与我的队友合作,同时也感谢老师的指导。
