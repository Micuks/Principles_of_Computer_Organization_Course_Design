通过这次课程设计,我认识到对于计算机科学专业课的学习光有理论课是不够的,更应该理论
与实践相结合。因为理论课上学习的内容都比较模糊,只能学到对计算机组成原理的大致印象;
而在课程设计课程上的具体实践,给了我一个亲手体会从设计到代码实现硬布线控制器的机会,
让我对无论是计算机组成原理中运算器,控制器,寄存器,存储器等部件,还是设计和实现的
流程,包括设计功能,流程图,译码表,编写伪代码,到编写实际代码的过程,都有了深刻
而清晰的认识。
\par
在这次课程设计中,在仅有6天的紧迫时间中,我从零开始对SyetemVerilog语言,以及其仿
真调试方法TestBench和用来仿真的Modelsim软件,利用网上的各种文档和示例进行了解和
自主学习,对硬件描述语言(VHD)的理解也从一开始类似C的高级语言到最终的每个语句,
每个块都对应一定的组合逻辑或时序逻辑电路。
\par
在一开始对硬件描述语言的理解不够到位的时候,我将时序逻辑部分的描述语句和组合逻辑部分
的描述语句都塞到一个always块当中用时钟驱动,导致在设计时定义为寄存器的变量,在编译后
就变成了锁存器。这也体现了我仅仅通过理论学习,对组合逻辑,时序逻辑,以及时钟在电路设计
中的重要作用没有深刻到位的理解;在50多个小时的代码编写过程中,我对硬件描述语言和时钟
的理解逐渐加深,同时,也意识到了组合逻辑部分硬件和时序逻辑部分在电路中为发挥作用各自
所发挥的功能以及发挥功能的方法,对数字逻辑,计算机组成原理等都有了亲身实践而深刻的理解。
\par
同时,通过这次课程设计还增强了我的创新和探索能力。在IAR的两个控制引脚没有给出的情况下,
在我们小组仔细讨论分析之后,找到了使用通用寄存器作为PC的镜像的方法来间接实现中断后PC值
的保存,以及中断返回。
\par
此外,通过这次课程设计还让我体会到了硬件亲身实践和理论学习以及软件仿真不同的地方。例如,
我在对硬件描述语言和组合逻辑、时序逻辑理解不够深刻的时候,将所有信号的变化都放在always
块中,虽然仿真结果和预想相同,但是放在\tec 上却总是出现和预期不相符的信号变化。
在经过仔细思考后,将组合逻辑和时序逻辑部分分离,将Quartus II的近百个warning全部消除,
使得代码的健壮性大为增强。再比如,在写好中断功能的代码后,我们发现按下PULSE信号后,
\tec 总是没有收到信号。苦苦调试一下午无果后,我在\tec 实验指导书中注意到,MF等
实验台信号需要用彩虹线将S7128芯片和实验台的时钟部分连接才能收到。在进行连接后,果然
正常收到了PULSE信号,进入了中断。
\par
总的来说,在一周的实验过程中,我们通过集合团队的力量,对设计、代码编写和调试分工合作,
实现了$1+1 > 2$的效果,有了不小收获。另一方面,老师提到的同学应当更加自信和有追求,
敢于探索的说法也引起了我深刻的反思。