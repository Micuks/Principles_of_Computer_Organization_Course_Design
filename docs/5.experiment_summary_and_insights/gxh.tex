在本次计算机组成原理课程设计,我和其他三位同学共同完成了硬连线控制器的顺序型,流水型的设计以及中断功能的设计。在本次实验中,因为我是和三个
实验班的同学进行组队,我认为我最大的收获就是学习到了别人对于实验认真负责的态度和努力钻研的精神。
\par  
在第一天,经过老师对于实验内容的讲解,我对于本次实验有了一个初步的认识。本次设计的理论基础来自于大二上所学的《数字逻辑》与本学期所学的《
计算机组成原理》,其中 TEC-8 的数据通路也在本学期的计组实验中有所掌握。经过讨论分析,我们小组决定用Verilog语言进行编写,由于之前在数字逻辑的
学习中我只学习了vhdl语言,因此遇到的第一个难题就是需要快速掌握Verilog语言的语法和编程细节,这也让我不禁感叹,完成任何一个难题都需要极强的自
学能力以及动脑思考的能力。而Verilog语言的一大难点就是程序是并行的,并且在编程时不能在同一模块对相同变量进行二次赋值这个问题困扰了我们很久,
最终我们决定,改变代码结构并通过译码表来实现组合逻辑。
\par 
在本次实验中,我认为最大的难点就是调试程序,这也让我认识到一个程序的代码不能只要求理想化的设备,在代码编写的过程中我们更应该注重实际的实验
设备,这就要求我们考虑设备的灵敏度等问题。我们在调试过程中遇到的另一个难题便是没有考虑显示设备,导致运行时出现错误。之后便是中断程序的设计
我们最开始的想法是将pc的值保存在存储器的固定位置上,然后这样就可以在一个寄存器中保存一个栈顶,并在存储器的位置中设计一个类似金丝雀值的数,
由于我们在顺序型的实际中耗费了很多时间,一次没有继续研究下去,实在是很可惜。
\par 
在本次课设的过程中,我第一次感受到了线下实验的魅力所在,以及程序在现实设备上运行成功时的喜悦。同时我认为线下做实验的的确确是比线上要强很多
,真的非常有助于团队之间的交流互相帮助,互相学习。同时,我也明白了一定要敢于去自己动手,并且再失败时想明白为什么要失败,而不是失败就直接放
弃之前的思路。我也明白了,再提出自己的想法之前一定要思考一下是否实际,是否可行。多一些自己的思考,不要执迷于依赖队友。我也明白了,要靠自己
的努力,而我之前不够努力,其他三个队友为什么厉害,需要我去反思。
